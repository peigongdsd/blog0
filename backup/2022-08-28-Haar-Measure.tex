--
title: Cheatsheet on Haar Measure
author: PL Lee
---


\documentclass[12pt,a4paper]{article}
\usepackage{amssymb}
\usepackage{amsthm}
\newtheorem{clm}{Claim}
\newtheorem{def}{Definition}
\newtheorem{rem}{Remark}
\newtheorem{lem}{Lemma}
\newtheorem{thm}{Theorem}
\newtheorem{exp}{Example}
\newtheorem{prop}{Proposition}

\begin{document}
This is a cheatsheet on topological group and it's corresponding Haar measure, in case of future reference.

\section{Some Topological and Catogorial Stuff}

\section{Topological Group}
A topological group $G$ is a group with topological structure, that is
\begin{itemize}
    \item The composition $\cdot: G\times G\rightarrow G$ is continuous on $G\times G$.
    \item The inverse $\ ^{-1}: G\rightarrow G$ is continuous.
\end{itemize}

Moreover, morphisms between topologial groups are those group morphisms which are continous respecting to the topological structure.

This implies
\begin{itemize}
    \item For every element $h\in G$, the left transform corresponding to $h$:
        \[\tau_h: g\mapsto hg\]
        is a homoemorphism. So do the inverse operation.
\end{itemize}

The topological structure of a group $G$ induces topological structures on it's kernels and images.
\begin{itemize}
    \item (right object)Group homomorphism $\pi$ from topological group $G$ to group $H$ induces topological structure on $H$. In this case, the derived topology on $H$ is the finest topology such that $\pi$ is continuous.
    \item (left object)Group homomorphism $\pi$ from group $H$ to topological group $G$ induces topological structure on $H$. In this case, the derived topology on $H$ is the roughest topology such that $\pi$ is continuous.
\end{itemize}
In these cases, the homomorphisms $\pi$ naturally lift to morphisms between topological groups. For any group $G$, The initial morphism $0\rightarrow G$ gives the discrete topology, and the final morphism $G\rightarrow 0$ gives the trivial topology.


Here are some tricks on topological groups.
\begin{prop}
    \begin{itemize}
        \item Let $H$ be a open subgroup of topological group $G$, then $H$ must be closed.
        \item Let $V$ be a open subset of topological group $G$, then for any subset $U\subset G$, $UV$ is open.
    \end{itemize}
\end{prop}
\begin{proof}
    Trivial because $H$ is the complementation of the union of all cosets of itself, which are open. The second one is also trivial.
\end{proof}
\begin{prop}
    For any open subset $V\subset G$ containing the unit $e\in G$, 
    \begin{itemize}
        \item There is a open subset $U\subset G$ s.t. \[U^{-1}U\subset V\]
    \end{itemize}
\end{prop}
\begin{proof}
    Due to the fact that group operations are continuous, there are $V_1$ and $V_2$ s.t. $V_1^{-1}V_2\in V$. Then, let $U=V_1\bigcap V_2$. Morover, we can let $V$ contains the unit.
\end{proof}
\begin{prop}
    Some asserts on images of topological group.
    \begin{itemize}
        \item Let $G$ be a topological group and $H\leq G$ be a closed subgroup. Then $G/H$ is Hausdorff.
        \item Let $G$ be a locally compact group and $H\leq G$. Then $G/H$ is locally compact.
    \end{itemize}
\end{prop}
\begin{proof}
    Firstly, let $xH, yH\in G/H$ be distinct. Then $yHx^{-1}=(yH)(x^{-1})\subset G$ is closed and doesn't contain the unit. Thus, we can construct a open subset $V\subset G$ containing the unit $e\in G$ s.t. $V^{-1}V\subset G\backslash yHx^{-1}$. In other words, this is $V^{-1}V\bigcap yHx^{-1}=\varnothing$, which implies that $VxH$ and $VyH$ are distinct. Due to the observation that $VxH$ and $VyH$ are open neighborhoods of $xH$ and $yH$ in $G/H$, the proposition is clear.
    \par For the second assertion, it suffices to show $H$ has a compact neighborhood in $G/H$. Let $U$ be a compact neighborhood of $e$ in $G$, and $V\subset G$ open s.t. $V^{-1}V\subset U$. Write the canonical morphism from $G$ to $G/H$ as $\pi$, then $\pi(V)$ is open due to the definition of topology on $G/H$. 
\end{proof}

\section{Haar Measure on Topological Groups}
For a locally compact Hausdorff space $X$, Riesz's famous representation theorem gives a connection between \emph{positive linear functionals} and \emph{positive Borel measures} on $C_c(X)$, which refers to all the continue functions from $X$ to, say, complex numbers with a compact support.
\begin{thm}[Riesz's Representation Theorem]
    Let $X$ be a locally compact Hausdorff space, and let $\Lambda$ be a positive linear functional on $C_c(X)$. Then there exists a \sigma-algebra $\mathfrak{M}$ on $X$ which contains all Borel sets in $X$, and there exists a unique positive measure $\mu$ on $\mathfrak{M}$ which represents $\Lambda$ in the sense that 
    \begin{itemize}
        \item $\Lambda f=\int_X f\,d\mu$ for every $f\in C_c(X)$;
    \end{itemize}
with the additional properties below:
    \begin{itemize}
        \item $\mu(K)<\infty$ for every compact $K$;
        \item $\mu(P)=inf\{\mu(V)|P\subset V,\ V\ is\ open\ in\ X\}$;
        \item $\mu(P)$ is either $\infty$ or $sup\{\mu(K)|K\subset P,\ K\ is\ compact\ in\ X\}$.
    \end{itemize}
\end{thm}
The proof is quite long and boring so we will just omit it. Note that the second additional property is called "outer regular", and the third property, if without the restriction of $\mu(p)\not=\infty$, is called "inner regular".


\end{document}

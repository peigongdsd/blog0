---
title: Noetherian Space and Dimension
author: PL Lee
---

\usepackage{amsmath}
\usepackage{amssymb}
\usepackage{amsthm}
\newtheorem{def}{Definition}
\newtheorem{thm}{Theorem}
\newtheorem{lem}{Lemma}
\newtheorem{prp}{Proposition}
\newtheorem{rem}{Remark}

\newcommand{\spec}[1]{\mathfrak{Spec}(#1)}

\begin{document}


\section{Zariaki Topology of Rings}
\begin{def}[Noetherian Space]
    A topological space is called Noetherian iff every desending chain of closed sets
    \[ L_1\supseteq L_2\supseteq\cdots \]
    will eventually stablize. That is, there is some $n$ s.t.
    \[ L_n=L_{n+1}=\cdots \]
\end{def}

\begin{thm}
    If a topological space $X$ is both Noetherian and Hausdorff, it must be finite and descrete.
\end{thm}
\begin{proof}
    Fix $p\in X$, pick a point other than $p$, say $q_1\in X$, and let $V_1$ and $W_1$ be distinct open sets containing $p$ and $q_1$ respectively. Then we see that $p\in(X\backslash W_1)=L_1$. If $p$ is the only element of $L_1$, it must also be the only point of $V_1$. Otherwise, pick another $q_2\in L_1$, and let $V_2$ and $W_2$ likewise. Then $p\in(L_1\backslash W_2)=L_2$, and $p\in V_1\bigcap V_2\subseteq L_2$. Likewise, if this operation eventually stops at step $t$, we will know that $p$ is the only point of the open set $V_1\bigcap V_2\bigcap\cdots\bigcap V_t$. Otherwise, repeating this operation and we will get a decending chain
    \[ L_1\supseteq L_2\supseteq\cdots \]
    Since $X$ is Noetherian, the chain must stablize, which implies there must be some $L_n=L_{n+1}$, which is impossible due to the selection of $q_{n+1}$. Thus we know, every point of space $X$ is open, hence $X$ is descrete. Apply the property of Noetherian again it'll be revealed that $X$ is finite.
\end{proof}

\begin{def}[Spectrum of a ring]
    For a certain ring $R$, the spectrum $\spec{R}$ is the set of prime ideals of $R$.
\end{def}
\par Fixe a ring $R$, there is an antitone Galois connection between the set of ideals of $R$ and the subsets of $\spec{R}$. To see this, let
\[ V(T) = \{p\in\spec{R}|T\subset p\} \]
and
\[ Z(S) = \bigcap_{p\in S}p \]
in which $V$ maps an ideal to a subset of $\spec{R}$, and $Z$ reversely. Under these conventions, the closure operation
\[ T\mapsto Z(V(T)) \] maps an ideal to it's radical. 
\begin{lem}
    For any ring $R$, there is a minimal prime ideal $p\in\spec{R}$.
\end{lem}
\begin{proof}
\end{proof}
\begin{proof}
\end{proof}

\par The spectrum of a ring $R$ can be made into a topological space, in the sence that a subset $S\subseteq\spec{R}$ is closed if and only if there is a $T\subseteq R$ such that $S=Z(T)$. In this sence, the closure of a subset $S\subseteq\spec{R}$ is $V(Z(R))$, identical to closure in the sence of Galois connection mentioned above. This is called the Zariski topology. Let
\[ D(f) = \{p\in\spec{R}|f\not\in p\} \]
Since $D(f)=\spec{R}\backslash V(\{f\})$ we know that $D$ maps $R$ to open subsets of $\spec{R}$. For any closed set $V(T)$, in which $T$ is nonempty, let $f\in T$, then we have $D(f)\bigcap V(T)=\varnothing$, which implies $\{D(f)|f\in R\}$ is a base of topological space $\spec{R}$. 
\begin{def}
    Let $X$ be a topological space.
    \begin{itemize}
        \item We say $X$ is irreducible if and only if $X$ is nonempty, and cannot be written into $X=T_1\bigcup T_2$, where $T_i$ are closed in $X$.
        \item We say $Z\subset X$ is an irreducible component of $X$ if $Z$ is a maximal irreducible subset of $X$.
    \end{itemize}
\end{def}

\section{Morphism and Localization}
Let $\varphi$ be a homomorphism from ring $R$ to ring $R'$, then there is a natural homoemorphism
\[ \spec{\varphi}:\spec{R'}\rightarrow\spec{R} \]
Where In other words, $\spec{\cdot}$ is a contravariant functor from category of rings to category of topological spaces.


\end{document}


 
